\documentclass[10pt,usenames,dvipsnames]{beamer}
\usepackage{swjtu}
\usepackage[]{biblatex}
%\setbeamertemplate{bibliography item}{\insertbiblabel}
\addbibresource{bibliography.bib}
\usepackage[spanish]{babel}
\usepackage{amsmath} 
\usepackage{amsfonts}
\usepackage{amssymb}
\usepackage{amsthm}
\usepackage{xcolor}
%\usepackage{mdframed}
\usepackage{inputenc}
\usefonttheme{professionalfonts}
%\usefonttheme{serif}
\usepackage{graphicx}
\setbeamercovered{transparent}
\usepackage{ragged2e}
\usepackage{circuitikz}
\justifying{

\title{\textbf{3. Overlapping Generations Model (OLG)}}
\subtitle{Advanced Macroeconomics}
\author[Daniel Kelly (Colmex)]{Daniel Kelly}
\institute[Colmex]{El Colegio de México}
\date{Winter 2024}

\begin{document}
%% Title page
%% Remove '[plain]' if you need a footline
\begin{frame}[plain]
    \titlepage
\end{frame}

{
\begin{frame}[plain]
    \frametitle{Outline}
    \tableofcontents
\end{frame}
}

\begin{frame}[plain]
\textbf{Reading:} Romer (2012), \textit{Advanced Macroeconomics}, Chapter 2: Part B.
\end{frame}

\section{Introduction}

\begin{frame}{Introduction}

\end{frame}

\section{The Diamond Overlaping Generations Model}



\subsection{Household Preferences}
\begin{frame}{Household Preferences}
Households that belong to cohort $t$ (born on $t$) have an \alert{additively separable} utility function between consumption in both periods given by:
\eq{U(c_{t},c_{t+1})=u(c_t)+\beta u(c_{t+1})}
\pause
The preferences for consumption on each period, summarized by the utility function $u(\cdot)$ (that is twice differentiable) and:
\begin{itemize}
\item \textbf{Risk Aversity}, so that $u(\cdot)$ is concave and, in particular $u''(x)<0$, $\forall x$.
\item \textbf{Strictly Monotonic}, so that $u(\cdot)$ is strictly increasing, and in particular $u'(x)>0$, $\forall x$.
\end{itemize}
\end{frame}

\begin{frame}
The \alert{discount factor} $\beta$, as usual, is defined as:
\eq{\beta\equiv\dfrac{1}{1+\rho}} 
\pause
Where $\rho$ is the \alert{discount rate}, that shows the differences in valuation of consumption for the individual across different periods:
\begin{itemize}
\item If $\rho>0$, the individual places a \green{greater weight on first period consumption} $\Rightarrow$ $\beta<1$.
\item If $\rho<0$, the individual places a \green{greater weight on second period consumption} $\Rightarrow$ $\beta>1$.
\end{itemize}
\bigskip
\pause
Furthermore, we assume that $\rho>-1$, so that the weight put on second period consumption is positive. \\
\end{frame}

\begin{frame}
A \alert{balanced growth path} (Backward Induction) requires the assumption that the individual period utility function $u(\cdot)$ takes the CRRA (\textit{Constant Relative Risk Aversion}) form:
\eq{U(c_{t},c_{t+1})=\dfrac{c_t^{1-\theta}}{1-\theta}+\beta\cdot \dfrac{c_{t+1}^{1-\theta}}{1-\theta}}
\pause
In regards to \alert{population dynamics}, the number of young people (this is, those who are born) at period $t$ is denoted $L_t$, and we assume that population grows at a constant rate $n$:
\eq{L_t=(1+n)L_{t-1}}
\end{frame}

\subsection{Capital and Production}
\begin{frame}{Capital and Production}
Output is \alert{homogeneous}, and can be used either for consumption or investment. Capital is owned by households and is rented out to firms.\\
\bigskip
The whole economy is characterized by three perfectly competitive markets:
\begin{enumerate}
\item Market for Output (in which there are zero profits).
\item Labor Market (in which labor earns exactly its marginal product).
\item Capital Market (in which the rental price of capital is equalized to its marginal product).
\end{enumerate}
\pause
\bigskip
We assume that there is \alert{full capital depreciation} to make the calculations easier (so that $\delta=1$).
\end{frame}

\begin{frame}
Production is characterized by an \green{aggregate neoclassical production function} with labor augmenting technological progress:
\eq{Y_t=F(K_t,A_tL_t)}
\pause
Technological progress, grows at a constant rate $g$ and its dynamic is described by the difference equation:
\eq{A_t=(1+g)A_{t-1}}
\end{frame}

\begin{frame}
Each household only works when they are young, and they provide, inelastically, 1 unit of labor. The wage in the labor market is equal to the marginal product of effective labor:
\eq{W_t=\pd{F(K_t,A_tL_t)}{A_tL_t}=\pd{F(K_t,A_tL_t)}{L_t}\cdot A_t=w_tA_t}
\end{frame}

\subsection{Solution of the Model}

\begin{frame}{Summary of the Setup}
\begin{itemize}
\item In period $t$, capital supplied by the \textbf{old} and labor from the \textbf{young} are combined to produce output.
\item The \textbf{old} consume the gains from capital and their existing wealth, then die and exit the model.
\item The \textbf{young} divide their capital between consumption and savings for the next period (when they become \textit{old}).
\item Capital stock in period $t+1$ is equal to the savings by the young on period $t$ times the amount of young households:
\eq{K_{t+1}=L_t(A_tw_t-c_t)}
\item Capital is then combined with the labor supply of the following generation to produce output in period $t+1$.
\end{itemize}
\end{frame}

\begin{frame}
\begin{center}
\textbf{Figure 3.1:} Structure of the Diamond Overlapping Generations Model \\
\bigskip
\resizebox{0.7\columnwidth}{!}{%
\begin{circuitikz}
\tikzstyle{every node}=[font=\normalsize]
\draw [->, >=Stealth] (1.5,14.25) -- (11.5,14.25);
\draw [short] (6.75,14.5) -- (6.75,14);
\draw [short] (2,14.5) -- (2,14);
\draw [short] (10.5,14.5) -- (10.5,14);
\node [font=\normalsize] at (2,14.75) {$t-1$};
\node [font=\normalsize] at (6.75,14.75) {$t$};
\node [font=\normalsize] at (10.5,14.75) {$t+1$};
\draw [, dashed] (1.25,12) rectangle  (7.5,11.25);
\node [font=\normalsize] at (1,13) {(...)};
\node [font=\normalsize] at (2,13) {Old};
\node [font=\normalsize] at (6.75,11.75) {Old};
\node [font=\normalsize] at (6.75,10.5) {Young};
\node [font=\normalsize] at (2,11.75) {Young};
\draw [dashed] (1.25,13.25) -- (2.5,13.25);
\draw [dashed] (2.5,13.25) -- (2.5,12.5);
\draw [dashed] (2.5,12.5) -- (1.25,12.5);
\node [font=\normalsize] at (10.5,10.5) {Old};
\draw [dashed] (10,9.5) -- (10,8.75);
\draw [dashed] (10,8.75) -- (11.25,8.75);
\draw [dashed] (10,9.5) -- (11.25,9.5);
\node [font=\normalsize] at (10.75,9.25) {Young};
\draw [, dashed] (5.75,10.75) rectangle  (11.25,10);
\draw [ color={rgb,255:red,255; green,38; blue,0} , rounded corners = 10.8, ] (5.5,9.75) rectangle (8,15.25);
\node [font=\normalsize] at (3.25,7.75) {\textbf{Young}};
\node [font=\normalsize] at (9.5,7.5) {\textbf{Old}};
\node [font=\normalsize] at (3.25,7.25) {- Income: $A_tw_t$};
\node [font=\normalsize] at (3.25,6.75) {- Consumption: $c_t$};
\node [font=\normalsize] at (3.25,6.25) {- Savings: $A_tw_t-c_t$};
\node [font=\normalsize] at (9.5,7) {- Consumption: };
\node [font=\normalsize] at (9.5,6.5) {$R_{t+1}\cdot K_{t+1} = R_t\cdot L_t(A_t-c_t)$};
\draw [rounded corners = 10.8] (0,16) rectangle (13.25,5);
\end{circuitikz}
%
}
\end{center}
\end{frame}

\begin{frame}{Problem of Firms}

\end{frame}

\section{Model with Log Utility and Cobb-Douglas Tech.}

\section{Policy Experiments}

\section{Capital Accumulation and Dynamic Inefficiency}

\section{Macroeconomics of Pensions}
\begin{frame}
sadas
\end{frame}

%% Final page
\begin{frame}[plain]
    \begin{picture}(0,0)
        \put(-21.5, -164){\includegraphics[width=\paperwidth]{Imágenes/final_page_bg.png}}
    \end{picture}
\end{frame}

\end{document}
