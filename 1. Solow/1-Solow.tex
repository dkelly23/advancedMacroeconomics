\chapter{The Solow Growth Model}

\section{Introduction}

\textbf{Reading:} Romer (2012), Advanced Macroeconomics, Chapters 1 \& 4.

\section{The Solow Growth Model}

On the Solow Model, we have a continous time dynamic, in which each variable is dated continously (at every instant of time). We will denote the value of $x$ at time $t$ as:
$$x(t) \text{ where } t\in \mathbb{R}$$

The difference between consecutive values of $x$, when the difference in time is very small can be just interpreted as the derivative of $x$ with respect to $t$:
\eq{\dot{x}}

\definition{Neoclassical Production Function}{
A production function is neoclassical if it satisfies the following 4 properties:
\begin{enumerate}
\item \textbf{Constant Returns to Scale}:
\eq{F(cK,cAL)=cF(K,AL)}

\item \textbf{Positive and Diminishing Returns to Inputs}:
\eq{
\begin{array}{ccccccc}
F_K>0 & F_L>0 \\
\\
F_{KK}<0 & F_{LL}<0
\end{array}
}
This is, holding constant the level of technology, each additional unit of a factor of production yields positive results, but this effect diminishes as the volume of input increases.

\item \textbf{The Inada Conditions}:
\eq{
\begin{array}{cccccc}
\displaystyle\lim_{K\to 0}F_K=\infty & \displaystyle\lim_{K\to \infty}F_K=0 \\
\\
\displaystyle\lim_{L\to 0}F_L=\infty & \displaystyle\lim_{L\to \infty}F_L=0 \\
\end{array}
}
This is, the marginal product of an input approaches zero (infinity) as the level of that input goes to infinity (zero). These are technical assumptions that ensure the behavior of the derivatives of the production function.

\item \textbf{Essensiality:}
\eq{
\begin{array}{cccc}
F(0,AL)=0 & F(K,0)=0
\end{array}
}

\end{enumerate}
}